\documentclass[a4paper,titlepage,twocolumn]{article}
\usepackage{fancyhdr}
% set 1-inch margins in the document
\usepackage[left=1in,right=1in,top=1.2in,bottom=1in]{geometry}
%\usepackage[left=0.2in,right=0.2in,top=0.5in,bottom=0.5in]{geometry}
\usepackage{lastpage}
% include this if you want to import graphics files with /includegraphics
\usepackage{graphicx}
\usepackage{lastpage}
\usepackage{pdfpages}
\usepackage[toc,page,title,titletoc,header]{appendix}
\usepackage[unicode]{hyperref}
%\newcommand{\chapter}[1]{\begin{center}\large\sffamily{#1}\end{center}}

\usepackage{titlesec}

%%%%%%%%%%%%%%%%%%%%%%%%%%%%%%%%%%%%%%%%%%%%%%%%%%%%%%
\usepackage{xeCJK}
%\usepackage{fontspec}
\setCJKmainfont[BoldFont=simhei.ttf]{simsun.ttf}
%\setCJKsansfont{simhei.ttf}
%\setCJKmonofont{simfang.ttf}

%\setCJKmainfont{Adobe Song Std}
%\setCJKmainfont[BoldFont=Adobe Heiti Std]{Adobe Song Std}
%%%%%%%%%%%%%%%%%%%%%%%%%%%%%%%%%%%%%%%%%%%%%%%%%%%%%%

\hypersetup{pdfauthor={周吕文}, 
            pdftitle={综合力学基础课知识要点整理}, 
            pdfsubject={中科院力学所博士生考试},
            pdfkeywords={综合力学基础},
            pdfproducer={XeLateX with hyperref},
            pdfcreator={Xelatex}}

\renewcommand\refname{\bf 参考文献}
\renewcommand\contentsname{\bf 目 \ \ \ 录}
\renewcommand\figurename{\bf 图}
\renewcommand\tablename{\bf 表}
\renewcommand\appendixname{\bf 附录}
\renewcommand{\appendixpagename}{附录}

\pagestyle{fancy}
\fancyhead{} % clear all fields 
\lhead{周吕文}
\chead{力学所(李家春等): 综合力学基础课} 
\rhead{第\ \thepage\ 页, ~共\ \protect\pageref{LastPage}\ 页} 

\begin{document}
\title{博士生综合力学基础课(2013年秋)\\期末复习材料  \vspace{-20pt}}
\author{{\Large 授课老师: 李家春等}\\ \vspace{50pt}\\ 周吕文\\ \href{mailto:zhou.lv.wen@gmail.com}{zhou.lv.wen@gmail.com}\\ \vspace{50pt}}
\date{中国科学院力学研究所\\2014年1月5日}
\maketitle

%\begin{center}\Large\bf 
%综合力学基础课期末复习材料\\
%周吕文
%\end{center}

\setcounter{page}{0}
\textbf{说明}: 本文档基本根据扣扣群中材料及上课的课件整理而成, 时间伧促, 难免有误, 发现问题请电邮我. 祝各位考试顺利.
\tableofcontents


\titleformat{\section}{\centering\Large\bfseries}{第\,\thesection\,章}{1em}{}
\newpage
\section{力学及其分支学科}

\subsection{力学学科特点及与其它学科区别}
力学的研究对象:
\begin{itemize}
\item 力学是研究力和运动基本规律的学科, 也是研究介质介质力学性质的学科.
\item 力学是研究机械运动的学科, 也研究与物理, 化学, 生命运动交叉的学科.
\item 力学是研究物质宏观性质的学科, 也研究物质细, 微观行为的学科.
\item 力学是一门经典的基础学科, 同时是有广泛应用和强大生命力的学科.
\end{itemize}
与其他学科的区别:
\begin{itemize}
\item 数学: 研究从实际抽象的数与形一般规律的科学.
\item 物理学: 研究热, 光, 电, 磁运动; 时空和物质结构的科学.
\item 天文学: 研究天体运动和演化规律的科学.
\item 化学: 研究与分子结构变化有关的现象.
\item 地球科学: 研究大气, 水体(海洋, 河流,地下水), 岩土体, 地壳和深部运动规律的科学.
\item 生命科学: 研究与新陈代谢, 遗传, 变异有关现象的科学.
\end{itemize}
力学的性质: 1. 基础科学之一; 2.工程设计的科学依据; 3.基础与工程间的桥梁; 4.带领工业前进.
\subsection{力学的发展}
\begin{itemize}
\item     经典力学:  17世纪, 牛顿力学\hphantom{力学}, Newton, Galierio\\
\hphantom{经典力学: }18世纪, 分析力学\hphantom{力学}, lagrange, Hamilton\\
\hphantom{经典力学: }19世纪, 连续介质力学, cauchy, Navier-Stockes.
\item 近代力学: 1900-1960. 我国的近代力学奠基人: 钱学森, 周培源, 郭永怀, 钱伟长.
\item 现代力学
\end{itemize}

\subsection{我国的近代力学奠基人}
\begin{itemize}
\item 钱学森: 卡门-钱公式; 跨声速相似率; 薄壳稳定性理论, 稀薄气体动力学, 物理力学,工程控制论, 星际航行概论
\item 郭永怀: 跨声速流理论, PLK方法, 高超声速空气动力学, 电磁流体力学, 爆炸力学
\item 周培源: 湍流模式理论, 相对论
\item 钱伟长: 板壳理论, 穿甲力学, 有限元, 应用数学, 中文输入
\end{itemize}

\subsection{经典力学体系}
\begin{itemize}
\item 牛顿力学体系: 牛顿第二定律, 欧氏空间, 能量函数, 无约束的力学系统.
\item 拉格朗日力学体系: 最小作用量原理, 微分流形, 拉格朗日函数, 理想约束的力学系统.
\item 哈密顿力学体系: 哈密顿正则方程, 辛空间, 哈密顿函数, 一般的动力学系统.
\end{itemize}

\subsection{力学学派及代表人物}
\begin{itemize}
\item 理论力学学派(苏俄): 安德罗诺夫, KBM, 鲁缅采夫, 阿诺德, 恰普雷金, 穆斯海里维奇, 拉夫伦捷夫, 柯尔莫果罗夫, 郎道.
\item 应用力学学派(欧美): 普朗特, 冯卡门, 泰勒, 钱学森. 技术科学思想: 从事重大工程技术中的关键科学问题.
\end{itemize}

%%%%%%%%%%%%%%%%%%%%%%%%%%%%%%%%%%%%%%%%%%%%%%%%%%%%%%%%%%%%%%%%%%%%%%%%%%%%%%%%%
\section{离散系统的动力学}
\subsection{牛顿力学的基本内容及意义.}
牛顿力学是以牛顿运动三定律及万有引力学为基础, 研究速度远小于光速的宏观物体运动规律的学科分支, 是经典力学的重要组成部分. 牛顿运动三定律及万有引力定律:
\begin{itemize}
\item 牛顿第一定律: 任何一个物体, 如果没有外力作用, 将保持它的静止状态或匀速直线运动状态, 直到外力或其它物体的作用迫使它改变这种状态为止. 这种保持物体运动状态的性质叫惯性, 故又称为惯性定律.
\item 牛顿第二定律: 物体的动量对时间的变化率与所受的外力成正比, 并和力的方向相同. 
\[
\frac{d m\vec{v}}{dt} = \vec{F}
\]
当质量为常数时, 又可表述为: 物体受外力作用时所获得的加速度大小与合外力大小成正比, 与物体的质量成反比, 加速度的方向与外力方向相同. 
\[
\vec{F} = m\vec{a}
\]
牛顿第二定律又称为运动定律, 它表明了力, 加速度与质量三者间的定量关系.
\item 牛顿第三定律: 两物体相互作用力总是大小相等, 方向相反, 且在同一条直线上. 故又称为作用, 反作用定律.

\item 万有引力定律: 任何两个物体间都存在着引力, 引力大小和两物体的质量的乘积成正比, 与两物体间的距离平方成反比, 力的作用线在两物体质心的连线上.
\[
F = \frac{Gm_1m_2}{r^2}
\]
\end{itemize}
意义: 牛顿力学概括了物体机械运动的基本客观规律, 绝大多数工程科学和物理学某些部分迄今还是建立在牛顿力学基础上的, 它是物理学发展史上一个划时代的成就.

\subsection{守恒律}
有关宏观世界某些物理量在运动过程中保持不变的基本定律:
\begin{itemize}
\item 动量守恒: 质点系不受外力或所受外力的合力为零时, 该质点系的总动量保持不变.
\item 能量守恒: 在一封闭系统中, 能量可以从一种形式转化为另一种形式, 或者从一个物体转移到另一个物体, 在转化和转移的过程中, 系统的总能量保持不变.
\item 质量守恒: 在任何孤立系统中, 不论发生何种变化或过程, 其总质量保持不变.
\end{itemize}

\subsection{三个宇宙速度及推导}

\begin{itemize}
\item 第一宇宙速度: 物体紧贴地球表面做匀速圆周运动的速度(人造卫星最小发射速度).

推导: 以地球半径$R_{\mathrm{e}}$为轨道半径, 质量为$m$的近地卫星, 其运动速度$v_1$即为第一宇宙速度. 由万有引力提供向心力得
\[
\frac{GM_{\mathrm{e}} m}{R_{\mathrm{e}}^2} = m\frac{v_1^2}{R_{\mathrm{e}}} \Longrightarrow v_1 = \sqrt{\frac{GM_{\mathrm{e}}}{R_{\mathrm{e}}}} = 7.9 \mathrm{km/s}
\]

\item 第二宇宙速度: 在地球上发射的物体完全摆脱地球引力的束缚所需的最小发射速度.

推导: 取无穷远处引力势能为零. 由物体和地球组成的系统机械能守恒得

\[
\frac{1}{2}mv_2^2 + \int_{R_{\mathrm{e}}}^\infty \frac{GM_{\mathrm{e}} m}{r^2} dr = 0 \Longrightarrow v_2 = \sqrt{\frac{2GM_{\mathrm{e}}}{R_{\mathrm{e}}}} = 11.2 \mathrm{km/s}
\]


\item 第三宇宙速度: 在地球上发射的物体完全摆脱太阳引力的束缚所需的最小发射速度. 

推导: 太阳质量为$M_{\mathrm{s}}$,太阳中心到地球中心的距离为$R_{\mathrm{se}}$, 类似于第二宇宙速度, 逃逸太阳需要的速度为
\[
v_{\mathrm{s}} = \sqrt{\frac{2GM_{\mathrm{s}}}{R_{\mathrm{se}}}}= 42.3 \mathrm{km/s}
\]
类似于第一宇宙速度, 地球绕太阳公转的运动速度为
\[
v_{\mathrm{e}} = \sqrt{\frac{GM_{\mathrm{s}}}{R_{\mathrm{se}}}} = 29.8 \mathrm{km/s}
\]
假设沿着地球公转的速度方向发射, 则相对地球的发射速度只需要$v'=v_{\mathrm{s}}-v_{\mathrm{e}}$, 但考虑到地球引力存在, 必须同时克服地球引力, 设第三宇宙速度为$v_3$
\[
\frac{1}{2}mv_3^2- \frac{1}{2}mv_2^2= \frac{1}{2}mv'^2 \Longrightarrow v_3 = \sqrt{v_2^2+v'^2} = 16.7 \mathrm{km/s}
\]

\end{itemize}


\subsection{相对论基本假设和结论}
狭义相对论的基本假设
\begin{itemize}
\item 相对性假定: 一切物理定律在所有惯性系中形式保持不变.
\item 光速不变假定: 光的传播与光源运动与否无关, 与所处惯性系无关; 与频率无关; 光速各向同性; (可以校准空间各点的时钟,确定同时性);
\item Lorenz 变换:
\[
x' = \frac{x-vt}{\sqrt{1-\frac{v^2}{c^2}}},\quad y'=y,\quad z'=z,\quad t' = \frac{t-vx/c^2}{\sqrt{1-\frac{v^2}{c^2}}}
\]
\end{itemize}
狭义相对论的结论: 长度变短, 时间膨胀, 多普勒频移, 质速关系, 质能关系.

\subsection{量子力学与牛顿力学的区别}
1.满足薛定锷方程; 2.分立的能谱; 3.不确定性: 测不准原理, 几率波; 4.微观世界的规律

%%%%%%%%%%%%%%%%%%%%%%%%%%%%%%%%%%%%%%%%%%%%%%%%%%%%%%%%%%%%%%%%%%%%%%%%%%%%%%%%%
\section{流体力学}

\subsection{流体的三种状态及特性}
\begin{itemize}
\item 三种状态: 液态, 气态, 等离体.
\item 流体介质的特性:1.无固定形状, 甚至无固定体积. 2.在剪切作用下发生流动.
\end{itemize}
 
\subsection{气体状态方程}
理想气体和非理想气体分别为
\[
PV = RT, \qquad
\bigg(P+\frac{a}{V^2}\bigg)(V-b) = RT
\]

\subsection{运动粘性系数表达的意义}
\begin{itemize}
\item 粘性系数$\nu$: 单位质量受到的剪应力与单位长度法向速度的变化之间的比值.
\item 动力学粘性系数$\mu$: 表示剪应力与单位长度法向速度的变化之间的比值.
\[
\tau = \mu\frac{du}{dy}, \qquad
\nu =\frac{\mu}{\rho}
\]
\end{itemize}

\subsection{流体本构与固体本构的区别}
\begin{itemize}
\item 流体力学的本构关系: 流体中的应力与速度梯度(速度关于空间得导数)有关. 应力张量是应变速率张量的线性函数
\[
\sigma_{ij} = -p\delta_{ij} + D_{ijkl}\frac{\partial u_k}{\partial x_l}
\]

\item 固体力学的本构关系: 固体中的应力与固体质点的位移梯度(速度关于空间得导数)有关.
\[
\sigma_{ij} = C_{ijkl}\varepsilon_{kl}, \quad \varepsilon_{ij} = \frac{1}{2}(u_{i,j}+u_{j,i})
\]
\end{itemize}

\subsection{牛顿流体与非牛顿流体}
牛顿流体是指, 流体中的应力与流体的应变率呈线性关系, 即对于各向同性流体有
\[
\tau_{i,j} = -p\delta_{i,j}+\lambda\frac{\partial u_i}{\partial x_i}\delta_{i,j} +\mu\bigg(\frac{\partial u_i}{\partial x_j} + \frac{\partial u_j}{\partial x_i}\bigg)
\]
不满足上述假设的流体称为\textbf{非牛顿流体}. 
\begin{itemize}
\item 非牛顿流体现象: 1.栓塞流动; 2.Weissenberg 效应; 3.挤出膨胀效应; 4.开口虹吸现象; 4.高分子溶液减阻(Thomson效应); 5.法罗伊斯, 林奎斯特效应(血流成分, 压降的管径影响).
\item 非牛顿流体分类: 1.广义牛顿流体(塑性流体; 拟塑性流体; 膨胀流体); 2. 触变流体(触稀流体; 震凝流体); 3.粘弹性介质.
\end{itemize}

\subsection{纳维-斯托克斯方程}
\[
\rho \frac{d\mathbf{v}}{dt}=-\nabla p + \rho\mathbf{F}+\mu\Delta v
\]
\subsection{宏观性质的微观解释}
\begin{itemize}
\item 压力: 压力是以气体分子对某个平面撞击所造成. 压力正比于单位体积内的气体分子动能平均值.
\[
p = \frac{1}{3}n m \overline{v^2}
\]

\item 温度: 温度是气体分子平均平动动能的量度, 是大量气体分子热运动的一种宏观表现. 温度与气体分子的速率平方成正比.
\[
T = \frac{m\overline{v^2}}{3k_B}
\]
\end{itemize}


\subsection{流体运动中的基本现象}
\begin{itemize}
\item 对流现象: 1. 强迫对流(物体运动, 静止物体置于流动中); 2.自然对流(大气海洋环流,热对流): 重力对流, 毛细对流; 3.混合对流.
\item 扩散现象: 
\begin{enumerate}
\item 动量扩散: 粘性(流体内部阻碍其相对流动的一种特性. 气体内摩擦或粘滞现象与分子热运动和分子间碰撞有直接联系. 气体内各部分定向动量不均匀, 分子热运动引起定向动量的输运过程);
\item 能量扩散:传热(气体内各部分温度不均匀, 由于分子的热运动使能量从温度高的区域输运到温度低的区域);
\item 质量扩散: 传质(当物体中密度不均匀时, 由于分子的热运动使粒子从密度较大处向密度较小处迁移的现象);
\end{enumerate}
\item 其它现象: 波动现象, 不稳定现象, 非平衡现象.
\end{itemize}


\subsection{常用的相似准则数(无量纲数)}
\begin{itemize}
\item 雷诺数$Re$: 惯性力与粘性力之比. 以英国工程师雷诺(O.Reynolds)命名, 是流体力学中最重要的相似准则数. 所有与粘性流动有关的模型实验都必须考虑.
\[
Re = \frac{\rho V l}{\mu}
\]
其中$l$为物体特征长度, $V$为特征速度, $\mu$运动粘性系数.

\item 弗劳德数$Fr$: 惯性力与重力之比. 以英国船舶设计师弗劳德(W.Froude)命名. 当模拟具有自由液面的液体流动时, 如水面船舶运动, 明渠流动等, $Fr$数是必须考虑的相似准则数.
\[
Fr = \frac{V}{\sqrt{gl}}
\]
其中$l$为船舶或明渠的特征长度, $V$为特征速度.

%\item 埃克曼数$Ek$: 粘性力与科里奥利力之比. 以海洋学家V.W.埃克曼而命名. 
%\[
%Ek = \frac{\nu}{\Omega L^2}
%\]
%$\Omega$为行星旋转的角速度, $L$特征长度, $\nu$为动力学粘度.

\item 欧拉数$Eu$: 压力或压差力与惯性力之比. 以瑞士数学家欧拉(L.Euler)命名. 当讨论流场中某点的特征压强或两点间的压强差时常用到.
\[
Eu = \frac{p}{\rho V^2}
\]


\item 斯特劳哈尔数$Sr$: 当地惯性力与迁移惯性力之比. 以德国物理学家斯特劳哈尔(V.Strouhal)命名, 他在研究风吹过金属丝发出鸣叫声时创立此数. 在研究不定常流动或脉动流时, $Sr$数成为重要的量纲为1的参数.
\[
Sr = \frac{l\omega}{V}
\]
其中$l$为物体特征长度, 如金属丝或圆柱的直径等, $\omega$为当地流体脉动圆频率.


\item 马赫数$Ma$: 惯性力与压缩力之比. 以奥地利物理学家马赫(E.Mach)命名. 当气体作高速流动时气体压缩性成为重要属性, $Ma$数用来描述流体压缩性的影响.
\[
Ma = \frac{V}{c}
\]
其中$c$为声速.

\item 牛顿数$Ne$: 外力与惯性力之比. 以英国物理学家牛顿(S.I.Newton)命名. 主要用于描
述由流体产生的阻力, 升力, 力矩和(动力机械的)功率等外力行为的影响.
\[
Ne = \frac{F}{\rho l^2 V}
\]
\end{itemize}


\subsection{无量纲数的物理意义(感觉更像是量纲分析的意义)}
\begin{itemize}
\item 量纲分析法又称为因次分析法, 是一种数学分析方法, 通过量纲分析, 可以正确的分析各变量之间的关系, 简化试验和成果整理, 所以量纲分析是我们分析流体运动的有力工具.

\item 自然科学中一种重要的研究方法, 它根据一切量所必须具有的形式来分析判断事物间数量关系所遵循的一般规律. 通过量纲分析可以检查反映物理现象规律的方程在计量方面是否正确, 甚至可提供寻找物理现象某些规律的线索.
\end{itemize}


\subsection{二相流的定义}
由多于一种物质状态的不相溶介质组成的流动系统称为多相流. 各相间存在明显的分界面.

%%%%%%%%%%%%%%%%%%%%%%%%%%%%%%%%%%%%%%%%%%%%%%%%%%%%%%%%%%%%%%%%%%%%%%%%%%%%%%%%%
\section{等离子体}
\subsection{等离子体定义, 性质及分类}
\begin{itemize}
\item 等离子体是处于一定温度下的电离气体, 物质的第四态. 它需满足以下三个条件:
\begin{enumerate}
\item 整体呈准电中性: $L\gg \lambda_D$. $L$为等离子体尺度, $\lambda_D$为德拜长度.
\item 集体效应起主要作用: $N_D \gg 1$. $N_D$为等离子体参量.
\item 响应时间小于碰撞时间.
\end{enumerate}
\item 等离子体具有独特的物理化学性质: 1.温度高, 粒子动能大; 2.具有导电性; 3.化学性质活泼, 易发生化学反应; 4.具
有发光特性.
\item 等离子体的分类: 1.热等离子体; 2.冷等离子体($T_e\gg T_h$); 3.大气压非平衡等离子体.
\end{itemize}

\subsection{德拜长度}
\begin{itemize}
\item 在等离子体中, 由于异性相吸, 一个带电粒子总是被一些带异号电荷的粒子包围, 这样, 带电粒子的静电场基本上只能在一定距离起作用, 在此距离外因周围异号电荷的屏蔽, 电场迅速消失, 这个距离称为德拜长度.
\item 物理意义: 静电作用的屏蔽半径; 局域性电荷分离的空间尺度.
\end{itemize}


\subsection{等离子体中的几个主要参数}
\begin{itemize}
\item 等离子体参量: $\displaystyle N_D=\frac{4}{3}\pi\lambda_D^3n_e$
\item 等离子体振荡频率: $\displaystyle \omega_{pe}=\sqrt{\frac{n_e e^2}{m_e \varepsilon_0}}$
\item 等离子体碰撞频率: $\displaystyle \nu=\sqrt{\frac{8kT_e}{\pi m_e}}\big(n_eQ_{em}+n_iQ_{ei}+n_aQ_{ea}\big)$
\item 德拜长度: $\displaystyle \lambda_D=\sqrt{\frac{\varepsilon_0kT_e}{n_ee^2}}$
\end{itemize}
其中$T_e$为电子温度, $n_e$为电子密度, $\varepsilon_0$为真空介电常数, $m_e$为电子质量, $Q_{ea}$为粒子的碰撞截面.

\subsection{等离子体的基本描述方法}
\begin{itemize}
\item 单粒子轨道描述法: 考虑单个粒子在电磁场中的运动, 适合稀薄等离子体的情况.
\item 磁流体力学描述法: 把等离子体当作导电流体来处理, 适合稠密等离子体的情况.
\item 等离子体动力论描述法: 用统计物理学的方法来处理.
\end{itemize}

\subsection{近平衡态输运}
\begin{itemize}
\item 偏离平衡态不远的实际过程, 因中间态不是平衡态而称为近平衡态的非平衡过程.
粘滞现象, 热传导现象和扩散现象即为近平衡态的非平衡过程.又因都有某一量的输运而称为近平衡态非平衡输运过程.

\item 等离子体中的四种主要输运过程: 等离子体中的输运现象是等离子体处于非平衡状态时发生的宏观现象.
当等离子体内部有密度, 温度, 速度等梯度或存在电场时, 将出现粒子流、能流、动量流或电流. 这些过程分别称为扩散(扩散系数$D$), 热导(热导率$\lambda$), 黏滞(黏性系数$\mu$)或电导(电导率$\sigma$)等. 它们可导致物理量在空间的传输.
\end{itemize}

%%%%%%%%%%%%%%%%%%%%%%%%%%%%%%%%%%%%%%%%%%%%%%%%%%%%%%%%%%%%%%%%%%%%%%%%%%%%%%%%%
\section{固体力学}

\subsection{应力张量与应变张量}
在服役状态下的结构或材料内部某点处某方向上单位面积上所受的内力为应力, 由应力引起的物体变形程度的力学度量是应变.
\begin{itemize}
\item 应力张量: 任意质点处的应力有6个独立分量,形成二阶张量:
\[
\sigma_{ij}=\left[\begin{array}{ccc}
\sigma_{11} & \sigma_{12} & \sigma_{13}\\
\sigma_{21} & \sigma_{22} & \sigma_{23}\\
\sigma_{31} & \sigma_{32} & \sigma_{33}
\end{array}\right],\quad\sigma_{ij}=\sigma_{ji}
\]

\item 应变张量: 任意质点处的应变有6个独立分量,形成二阶张量:
\[
\varepsilon_{ij}=\left[\begin{array}{ccc}
\varepsilon_{11} & \varepsilon_{12} & \varepsilon_{13}\\
\varepsilon_{21} & \varepsilon_{22} & \varepsilon_{23}\\
\varepsilon_{31} & \varepsilon_{32} & \varepsilon_{33}
\end{array}\right],\quad\varepsilon_{ij}=\varepsilon_{ji}
\]
\end{itemize}

\subsection{胡克定律}
描述线弹性应力-应变关系的定律, 应力$\sigma$与应变$\varepsilon$成正比, 比例常数为弹性常数(杨氏模量$E$)
\[
\sigma = E\varepsilon
\]
广义的胡克定律可表示为:
\[
\sigma_{ij} = C_{ijkl}\varepsilon_{kl}
\]

\subsection{屈服准则}
\begin{itemize}
\item 单向应力: 受力物体内质点处于单向应力状态时, 只要单向应力大到材料的屈服点时, 则
该质点开始由弹性状态进入塑性状态, 即处于屈服.
\item 多向应力: 受力物体内质点处于多向应力状态时, 必须同时考虑所有的应力分量. 在一定
的变形条件(变形温度, 变形速度等)下, 只有当各应力分量之间符合一定关系时, 质点才开始进入塑性状态, 这种关系称为屈服准则, 也称塑性条件. 它是描述受力物体中不同应力状态下的质点进入塑性状态并使塑性变形继续进行所必须遵守的力学条件.
\item 屈服准则是求解塑性成形问题必要的补充方程.
\end{itemize}

\subsection{蠕变和松弛}
\begin{itemize}
\item 蠕变: 材料在恒应力作用下变形随时间增大的变形过程称为蠕变, 是由分子或原子重新排列引起的. 蠕变过程中材料的柔度(模量的倒数)逐渐增大. 以应力为输入量而求应变响应者为蠕变.

\item 松弛: 材料在固定变形下应力随时间减小的过程称为松弛. 材料的模量(松弛模量)逐渐减小. 以应变为输入量而求应力响应者为松弛.

\end{itemize}


\subsection{线弹性, 弹塑性, 粘弹性本构模型}
\begin{itemize}
\item 线弹性:应力与应变成正比.
\item 塑性本构关系: 含``内变量''并与热相关.
\item 粘弹性本构关系(流变学): 材料机械性能与时间相关.
\end{itemize}

\subsection{基本方程}
\begin{itemize}
\item 平衡条件: $\sigma_{ij,j} + f_i = 0$
\item 几何方程: $\varepsilon_{ij}=\frac{1}{2}(u_{i,j}+u_{j,i})$
\item 本构方程: $\sigma_{ij} = C_{ijkl}\varepsilon_{kl}$
\end{itemize}

\subsection{复合材料}
\begin{itemize}
\item 定义: 两个及以上不同性质材料复合而成的材料. 
\item 特点: 1. 非均质性(纤维和基体之间界面层以及层间性能突变); 2.各向异性(单层板为正交异性); 3.横向性能比纵向性能差, 层合板横向剪切模量低; 4.层间应力有奇异性; 5.混杂效应.
\item 基本假定: 纤维与基体都是均质线弹性; 基体是各向同性; 纤维考虑其各向异性(以碳纤维为例)
\end{itemize}


\subsection{弹性力学一般定理}
\begin{itemize}
\item 功能互等定理: 弹性体内的应变能等于变形过程中外力所做之功.
\item 圣文南原理: 一组作用力可用另一组等效力代替, 在较远处应力分布不受影响.
\item 功的互易(等)定理: 设两组力(包括体力与面力)作用在同一物体上, 第一组力对第二组力的相应位移所做之功, 等于第二组力对第一组力相应的位移所做之功.
\end{itemize}

\subsection{能量原理}
热力学第一定律: 系统所吸收的热量$Q$等于外力功$A$加上系统内能的增量$\Delta U$.
\[
Q=\Delta U + A
\]

\subsection{虚位移, 最小势能, 最小余能及广义变分原理}
\begin{itemize}
\item 虚位移原理: 弹性体在平衡状态下, 外力对虚位移所做的功(称为虚功)等于虚位移所引起的弹性体应变能的增量.
\item 最小势能原理: 在一切变形可能状态中, 真实状态的总势能最小.
\item 最小余能原理: 在一切静力可能状态中, 真实状态的总余能最小.
\item 广义变分原理: 弹性力学的解必须满足广义势能变分为零的条件
\end{itemize}


%%%%%%%%%%%%%%%%%%%%%%%%%%%%%%%%%%%%%%%%%%%%%%%%%%%%%%%%%%%%%%%%%%%%%%%%%%%%%%%%%
\section{损伤力学}

\subsection{损伤力学定义}
\begin{itemize}
\item 损伤: 微裂纹, 微孔洞和局域材料性能劣化的总称, 是与完好无损的相对.
\item 损伤力学: 研究材料损伤机理和具体损伤种类(微裂纹或孔洞), 分布, 取向及其对材料断裂的影响.
\end{itemize}

\subsection{率效应}
材料所受到的应力随着材料性能的变化不断增大.


%%%%%%%%%%%%%%%%%%%%%%%%%%%%%%%%%%%%%%%%%%%%%%%%%%%%%%%%%%%%%%%%%%%%%%%%%%%%%%%%%
\section{断裂力学}
\subsection{断裂力学定义}
\begin{itemize}
\item 断裂: 由于裂纹或损伤的扩展而引起的整体分离.
\item 断裂力学: 研究含裂纹物体的应力场与位移场, 断裂强度与裂纹扩展规律. 研究任务为研究含裂纹物体是否断裂与测定材料的抗断裂性能.
\item 损伤力学是断裂力学前兆的研究.
\end{itemize}

\subsection{断裂准则}
\begin{itemize}
\item $J$准则: 弹塑性断裂准则, 沿裂纹尖端$J$积分守恒(J是能量释放率, 能量/面积), $J<J_c$时不会发生断裂.
\[
J = \int_\Gamma W dx_2 - T_i\frac{\partial u_i}{\partial x_l}ds
\]

\item $K$准则: 应力强度因子$K$准则, 当 $K<K_c$不会发生断裂.
\[
K_{\mathrm{I}} = \lim_{r\rightarrow 0}\sqrt{2\pi r}\sigma_{22}(r,\theta)\Big|_{\theta=0}
\]
\[
K_{\mathrm{II}} = \lim_{r\rightarrow 0}\sqrt{2\pi r}\sigma_{12}(r,\theta)\Big|_{\theta=0}
\]
\[
K_{\mathrm{III}} = \lim_{r\rightarrow 0}\sqrt{2\pi r}\sigma_{23}(r,\theta)\Big|_{\theta=0}
\]

\item $K$准则与$J$准则的区别: $K$准则是在线弹性假设下得到的, 适用于脆性材料, 在破坏前没有非线性应变; $J$准则适用于非线性的应变关系, 在破坏前发生屈服. 退化到线弹性, 两者相同.

\end{itemize}


\subsection{裂纹尖端渐近奇异场的意义和特征}
\begin{itemize}
\item 裂纹尖端的应力场正比于$1/\sqrt{r}$, 当$r\rightarrow 0$时它们趋于无穷, 这一性质称为应力具有奇异型;
\item 特征: 应力场在裂纹尖端邻域的强度可由一个仅依赖裂纹几何和载荷条件单一因子去表征.
\item 意义: 把断裂力学问题归结为一个综合的参数, 应力强度因子.
\end{itemize}

\subsection{应力强度因子的概念}
\begin{itemize}
\item 裂纹尖端应力场中仅依赖载荷条件的因子表示, 对I型, II型, III型裂纹分别表示为$K_{\mathrm{I}}=\sqrt{\pi a}p$, $K_{\mathrm{I}}=\sqrt{\pi a}\tau_1$, $K_{\mathrm{I}}=\sqrt{\pi a}\tau_2$, 其中$a$为裂纹长度, $p$, $\tau_1$, $\tau_2$为外载.

\item 应力强度因子实际上是能量释放率开方, 它把断裂力学问题归结为一个综合的参数.
\end{itemize}

\subsection{线弹性断裂准则对起裂和扩展的表述}
当应力强度因子$K$达到某临界值$K_c$时, 对理想脆性材料来说裂纹开始扩展, 对金属等韧性材料来说,当$K=K_c$时, 由于裂纹尖端出现塑性区, 裂纹开始起裂, 并不明显扩展,当$k$继续增大到$K_{ss}$时, 裂纹开始定常扩展.

\subsection{疲劳裂纹扩展准则的表述}
具有初始裂纹的结构, 即使裂纹未达到失稳扩展的临界尺寸, 但在交变应力下裂纹将会逐渐扩展导致疲劳破坏, 其破坏判据为: $\Delta K<(\Delta K)_{th}$, $\Delta K = K_{\max}-K_{\min}$. 可由paris公式$da/dN=C(\Delta K)^m$求得, $(\Delta K)_{th}$为门槛值.

\subsection{弹塑性断裂,对$J$控制参数的约束条件}
应力和应变单值对应, 小应变; 无体力.

\subsection{CTOD弹塑性准则}
当裂纹尖端张开位移达到某临界值时, 裂纹既要开裂.


%%%%%%%%%%%%%%%%%%%%%%%%%%%%%%%%%%%%%%%%%%%%%%%%%%%%%%%%%%%%%%%%%%%%%%%%%%%%%%%%%
\section{岩土力学}
\subsection{土的特点}
1.松散性: 多孔介质; 2.三相组成: 固体颗粒, 水, 气体; 3.自然变异性.


\subsection{有效应力原理}
\begin{itemize}
\item 饱和土承受的总压力为有效压力与空隙水压之和$\sigma = \sigma'+u$.
\item 土的变形(压缩)和强度的变化都只取决于有效应力的变化.
\end{itemize}

\subsection{达西定律}
\begin{itemize}
\item 达西定律: 渗流速度与水力坡度一次方成正比($k$为渗透系数):
\[
Q = kAi\quad \textrm{或}\quad  v=\frac{Q}{A}=ki
\]
\item 适用范围:达西定律不适用于堆石土, 致密土.
\end{itemize}

\subsection{反映细粒土结构特性的两种性质}
\begin{itemize}
\item 灵敏度: 原状土强度与重塑土强度之比(挪威黏土).
\item 触变性: 土因重塑而软化, 后因静置而逐渐硬化, 强度有所恢复的性质.
\end{itemize}


\subsection{临界水力坡降}
土体发生浮起或破坏时的临界状态时的水力坡降.

\subsection{压缩和固结}
\begin{itemize}
\item 压缩: 土的体积随有效应力的变化
\item 固结: 土体受力时,部分水量从土中排出, 外加压力相应地从孔隙水(与气)传递到骨架上, 直至变形达到稳定的过程.
\end{itemize}

\subsection{太沙基固结理论基本假设}
\begin{itemize}
\item 均质饱和
\item 土颗粒和孔隙水不可压
\item 单向渗流和压缩
\item 达西定律成立且渗透系数k为常数
\item 孔隙比的变化与有效应力的变化成正比,且压缩系数$a$保持不变
\item 外荷载一次瞬时施加
\end{itemize}

\subsection{与金属材料不同的基本力学特点}
\begin{itemize}
\item 压硬性抗剪强度和刚度随压应力的增大而增大
\item 等压屈服特性在各向等压作用下能产生塑性体变,出现屈服
\item 剪胀性体应变还与剪应力有关
\item 与应力路径的依赖性
\item 具有应变软化性质
\end{itemize}

%%%%%%%%%%%%%%%%%%%%%%%%%%%%%%%%%%%%%%%%%%%%%%%%%%%%%%%%%%%%%%%%%%%%%%%%%%%%%%%%%
\section{生物力学}
\subsection{细胞力学模型}
研究细胞的力学模型主要是连续介质模型, 连续介质模型把细胞看成是由具有某种连续材料性质的物质组成, 并且从实验中可以得到相应材料模型的本构关系及相关参数.
\begin{itemize}
\item 固体模型: 将整个细胞假设成均匀的且不含明显皮层的不可压缩弹性或者粘弹性固体.
\item 液滴模型: 细胞内部看成均匀同一的牛顿粘性流体, 而细胞皮层是各向同性粘性流体层, 该皮层只具有张力.
\end{itemize}


\subsection{测量细胞的实验方法}
施加已知的外力, 估计细胞的变型. 如微管吮吸.


%%%%%%%%%%%%%%%%%%%%%%%%%%%%%%%%%%%%%%%%%%%%%%%%%%%%%%%%%%%%%%%%%%%%%%%%%%%%%%%%%
\section{物理力学}

\subsection{物理力学的基本思想}
从物质的微观结构, 以及其组成粒子的运动和相互作用的规律出发, 建立适当的物理模型, 通过统计, 分析, 演绎等手段推导, 估算, 预测介质和材料的宏观性质.

\subsection{系综}
相应于同一个宏观态, 体系可能有大量甚至无限的微观态. 统计系综的概念对于由大量粒子组成的体系的描述是十分重要的. 一个系踪是全同体系的大量不同微观态的集合组成, 系踪的每个成员具有相同的宏观外部参数, 如固定的数密度, 温度等. 体系可用相空间$\Gamma$
的代表点 $\Gamma(q,p)$表示, 随着时间的变化, 代表点会在相空间中运动.
\[
\dot{q_i} = \frac{\partial H(q_i,p_i)}{\partial p_i},\quad 
\dot{p_i}=-\frac{\partial H(q_i,p_i)}{\partial q_i}
\]
系统分类
\begin{itemize}
\item 微正则系综: 孤立系统, 与外界既无能量又无粒子交换, $E$, $N$, $V$(能量,数密度,体积)恒定.
\item 正则系综: 封闭系统, 与大热源热接触平衡的恒温系统. 可以和大热源交换能量但粒子数固定的封闭系统. $N$, $V$, $T$(温度)恒定.
\item 巨正则系综: 开放系统, 与大热源大粒子源热接触平衡. 可以和大热源交换能量和粒子数的系统为开放系统. $V$, $T$, $\mu$(化学势)恒定.
\end{itemize}


\subsection{DLVO 理论}
\begin{itemize}
\item 溶胶粒子间的相互吸引力是范得华力, 主要为色散力. 胶粒中含有大量分子, 所以胶粒间的引力是各分子所贡献的总和, 这种引力与距离的三次方成反比, 是一种远程作用力.
\item 溶胶粒子间的排斥是由于带电胶粒所具有的相同电荷之间的斥力, 其大小取决于粒子电荷数目和相互距离.
\item 溶胶的稳定性取决于胶粒间吸引能和排斥能的总效应. 当粒子相互靠近时, 如果粒子间的引力大于粒子间的排斥力, 则溶胶发生聚沉, 是不稳定的; 反之, 溶胶是稳定的. 当粒子相互聚集在一起时, 必须克服一定的能垒.
\end{itemize}

\subsection{由微观量求宏观量的两种方法}
\begin{itemize}
\item 系综平均方法: 确定与宏观量$F$对应的微观量在某一微观态时的值-确定系统在微观态I停留的几率-对宏观条件允许的一切可能的微观态积分或求和得到-更基本, 是普适性的方法-计算机模拟中常用.

\item 统计热力学方法: 统计热力学方法是先求出系统的配分函数, 并求与之对应的特性函数, 再用求偏微商的简单运算求得系统的全部热力学量.
\end{itemize}

\subsection{分子平均自由程与平均碰撞频率}
\begin{itemize}
\item 分子平均自由程: 气体分子在相邻两次碰撞间飞行的平均路程.
\item 平均碰撞频率: 单位时间内一个气体分子与其它分子碰撞的平均次数.
\end{itemize}


\subsection{分子间力分类}
\begin{itemize}
\item 色散力(伦敦力): 由于分子内部的电子在不断地运动, 原子核在不断地振动, 使分子的正、负电荷重心不断发生瞬间相对位移, 从而产生瞬间偶极。瞬间偶极又可诱使邻近的分子极化,因此非极性分子之间可靠瞬间偶极相互吸引产生分子间作用力,由于从量子力学导出的这种力的理论公式与光的色散公式相似, 因此把这种力称为色散力(dispersion force). 虽然瞬间偶极存在的时间很短, 但是不断地重复发生,又不断地相互诱导和吸引,因此色散力始终存在。任何分子都有不断运动的电子和不停振动的原子核, 都会不断产生瞬间偶极, 所以色散力存在于各种分子之间.

\item 偶极-偶极相互作用: 包括极性分子和非极性分子间-诱导力和具有永久偶极的极性分子间-取向力.
由极性分子的永久偶极与非极性分子所产生的诱导偶极之间的相互作用力称为诱导力.

\item 氢键: 极性分子共价键中由氢和$F$,$O$或$N$组成的.
\end{itemize}

\subsection{蒙特卡罗(Monte Carlo, MC)}
\begin{itemize}
\item 蒙特卡罗(Monte Carlo)方法是指使用随机数(或伪随机数)来解决问题的方法.
\item 基本思想: 当所求解问题是某种随机事件出现的概率, 或者是某个随机变量的期望值时, 通过某种``实验''的方法, 以这种事件出现的频率估计这一随机事件的概率, 或者得到这个随机变量的某些数字特征, 并将其作为问题的解.
\end{itemize}

\subsection{分子动力学(Molecular Dynamics, MD)}
\begin{itemize}
\item 分子动力学方法不是用分布函数, 而是直接模拟多个微观粒子的运动和相互作用, 在相空间产生相应的分布.
\item 对$N$个相互作用粒子, 就其坐标和速度, 用差分法求解经典耦合运动方程. 得到在一段时间中体系在相空间的轨迹.
\item 分子动力学主要的优点不仅能提供平衡态的物理量, 也可模拟非平衡过程.
\end{itemize}


\subsection{布朗动力学(Brownian dynamics, BD)}
\begin{itemize}
\item MD的限制: 当流体中包含比原子分子大得多的粒子时(如血球,胶体粒子), 它们在空间和时间的尺度
上的要求与原子分子的相差巨大. 溶剂分子运动很快, 需要很小的时间步长. 如果我们只关
心大粒子的行为, 就需要很长的MD模拟时间. 

\item 如果我们对溶剂分子的运动细节并不关心, 它们的贡献用某种平均的效应(粘滞力和随机力)来代替. 因此在布朗动力学模拟中牛顿运动方程被朗之万方程代替:
\[
\frac{dp}{dt} = F(t)+\gamma p(t) + R'(t)
\]
其中$\gamma$为摩擦系数, $R'(t)$为代表布朗运动的随机项, 与MD相似, 布朗动力学模拟在$t$时刻用
差分求出$t+\Delta t$时刻的位置和力. 通过对上式一系列时间步长积分, 得出体系在相空间的轨
迹. 并对感兴趣的物理量进行系踪平均.

\item 优点: 由于略去了不关心的与溶剂分子有关的快运动信息, 只关注大粒子的运动情况,
大幅度地减少了体系的自由度. 因而BD模拟的时间步长可以比MD的大几个量级.
计算效率大大提高. 用随机力代替快运动自由度的结果使能量和动量不再守恒, 因而体系的宏观行为就不再是流
体力学的了.

\item 缺点: 一个溶质大分子通过溶剂对另一个大分子的影响被忽略, 即溶质粒子对溶剂组织结构等影响及其对溶质的反作用被忽略.
\end{itemize}


\section{冲击动力学}
\subsection{冲击定义与分类}
\begin{itemize}
\item 定义: 动能与势能之间相互转化的现象. 惯性: 可压缩性(恢复力)效应.
\item 分类: 1.一般冲击(低频$\ll (E/\rho)^2/l$, 物体各处发生同步振动); 2.高速冲击(高频$\gg (E/\rho)^2/l$, 引起波动及其后效应).
\end{itemize}

\subsection{流体中的连续波和冲击波}
\begin{itemize}
\item 有限幅度波: 绝热过程, 波速$v\pm c$, 声速$c=\big(\partial p/\partial \rho\big)_s^{1/2}$. $\Delta p=\rho_c\Delta v$, $\rho_c$为波阻抗.

\item 小扰动声波: 若$v=0$, 波速$c_0=\big(\partial p/\partial \rho\big)_s^{1/2}\big|_{p_0,\rho_0}$. 理想气体$c_0=\big(\gamma p_0/\rho_0\big)_s^{1/2}=(\gamma RT_0)^{1/2}$.

\item 间断击波: 后面压缩波追赶前面压缩波. 击波波速$U$: $(v+c)_{\textrm{\tiny 击波前方}}<U<(v+c)_{\textrm{\tiny 击波后方}}$.
%\item 点源强爆炸波: $r_s(t)=C\cdot (E/\rho_0)^{1/5}\cdot t^{2/5}$. $E$为爆炸能.
\end{itemize}

\subsection{弹性体中的体波, 面波和波导中的波}
\begin{itemize}
\item 体波: 弹性体内一点的扰动经有限时间传播到有限体积. 非色散(波速与波长无关). 纵波: 传播体积变化; 横波: 传播畸变.
\item 面波: 弹性体表面(界面)上一点的扰动经有限时间传播到有限面积. Rayleigh表面波: $c_R<c_s<c_d$, $c_s$为横波, $c_d$为纵波.
\item 波导中的波: 沿波导(一个方向的尺寸比其它方向小得多)长的方向上传播的波. 特点: 色散. 
\end{itemize}

\subsection{流体弹塑性体}
\begin{itemize}
\item 兼有流体性质和固体弹塑性的连续介质.
\item 应力和应变由体变和畸变两部分: 1.体变服从``状态方程''; 2.畸变服从``弹塑性应力-应变关系''.
\item 应变增量有弹性和塑性两部分: 1.弹性服从胡克定律; 2.塑性服从屈服准则和屈服后的塑性关系.
\item 本构含体积变化和畸变.
\item 无量纲参数: $p/Y$或$\rho v^2/Y$, 惯性与强度之比.
\end{itemize}
\newpage
\end{document}

